\documentclass{acm_proc_article-sp}
\pagenumbering{arabic}

\usepackage{epstopdf}
\usepackage{etoolbox}
\usepackage{listings}

\usepackage{multirow}
\usepackage{caption}
\usepackage{epstopdf}
\usepackage{float}
\usepackage[hyphens]{url}
\usepackage{etoolbox}
\usepackage{color}
\usepackage{listings}
\usepackage{lscape}
\usepackage{enumitem}
\lstset{ %
language={[x86masm]Assembler},                % choose the language of the code
basicstyle=\footnotesize,       % the size of the fonts that are used for the code
numbers=left,                   % where to put the line-numbers
numberstyle=\footnotesize,      % the size of the fonts that are used for the line-numbers
stepnumber=1,                   % the step between two line-numbers. If it is 1 each line will be numbered
numbersep=5pt,                  % how far the line-n/umbers are from the code
backgroundcolor=\color{white},  % choose the background color. You must add \usepackage{color}
showspaces=false,               % show spaces adding particular underscores
showstringspaces=false,         % underline spaces within strings
showtabs=false,                 % show tabs within strings adding particular underscores
frame=single,           % adds a frame around the code
tabsize=2,          % sets default tabsize to 2 spaces
captionpos=b,           % sets the caption-position to bottom
breaklines=true,        % sets automatic line breaking
breakatwhitespace=false,    % sets if automatic breaks should only happen at whitespace
escapeinside={\%*}{*)}          % if you want to add a comment within your code
}

\makeatletter
\patchcmd{\maketitle}{\@copyrightspace}{}{}{}
\makeatother


\begin{document}

\title{COS 597D Project - Mobile vs Traditional Web Tracking\\(FourthPartyMobile)}
%
% You need the command \numberofauthors to handle the 'placement
% and alignment' of the authors beneath the title.
%
% For aesthetic reasons, we recommend 'three authors at a time'
% i.e. three 'name/affiliation blocks' be placed beneath the title.
%
% NOTE: You are NOT restricted in how many 'rows' of
% "name/affiliations" may appear. We just ask that you restrict
% the number of 'columns' to three.
%
% Because of the available 'opening page real-estate'
% we ask you to refrain from putting more than six authors
% (two rows with three columns) beneath the article title.
% More than six makes the first-page appear very cluttered indeed.
%
% Use the \alignauthor commands to handle the names
% and affiliations for an 'aesthetic maximum' of six authors.
% Add names, affiliations, addresses for
% the seventh etc. author(s) as the argument for the
% \additionalauthors command.
% These 'additional authors' will be output/set for you
% without further effort on your part as the last section in
% the body of your article BEFORE References or any Appendices.

\numberofauthors{1} %  in this sample file, there are a *total*
% of EIGHT authors. SIX appear on the 'first-page' (for formatting
% reasons) and the remaining two appear in the \additionalauthors section.
%
\author{
% You can go ahead and credit any number of authors here,
% e.g. one 'row of three' or two rows (consisting of one row of three
% and a second row of one, two or three).
%
% The command \alignauthor (no curly braces needed) should
% precede each author name, affiliation/snail-mail address and
% e-mail address. Additionally, tag each line of
% affiliation/address with \affaddr, and tag the
% e-mail address with \email.
%
% 1st. author
\alignauthor
Marcela Melara, Christian Eubank and Diego Perez Botero \\ 
       \email{\{melara, cge, diegop\}@princeton.edu}
}

\maketitle

\begin{abstract}
INSERT ABSTRACT HERE??
\end{abstract}

\category{H.3.5}{Information Systems}{Information Storage and Retrieval}[Web-based services]
\category{K.4.1}{Computing Milieux}{Computers and Society}[Privacy]

\terms{Documentation, Measurement, Security}

\keywords{FourthParty, Web crawling, cookies, privacy policy, ...}

\section{Introduction}
DIEGO

We wish to automate the detection of third-party tracking mechanisms while browsing the web on a mobile device. To this end, we will adopt the FourthParty\footnote{http://www.fourthparty.info} project's approach and instrument a popular open-source mobile browser (i.e. Firefox) to be used as an enhanced web crawler. This enables us to log realistic end-user interactions (e.g. execution of embedded scripts) as opposed to just downloading each web page's static content, which is what traditional web crawlers do.

The mobile web crawler is not our main objective for this project, but rather the tool that we will use to collect valuable information in order to conduct our comparison between the Mobile and Traditional third-party tracking ecosystems and their practices.


\section{Background and Motivation}

\section{Related Work}
MARCELA

\section{Implementation}
DIEGO

\subsection{Challenges}
Mobile application development poses a variety of challenges that will need to be addressed for a mobile web crawler to be materialized:

\begin{itemize}
\item Mobile devices have limited amounts of RAM, so applications should not rely on large data structures stored in main memory.

\item Security permissions in mobile devices are strict, which means that writing data into persistent memory is not always an option.

\item Processing power in mobile devices is limited, so computationally intensive procedures, such as parsing a web page, should be delegated to an external entity.

\item Mobile network bandwidth is a limited resource, so large data transfers should be avoided.

\item Battery life must be preserved as much as possible by a mobile application if it is being aimed towards the general public.
\end{itemize}


\subsection{Mobile Web Crawler's Architecture}

FourthPartyMobile's architecture (see Figure \ref{fig:component_diagram}) delegates most of the computation and storage to a supporting server, limiting the mobile device’s responsibilities to fetching one website at a time and generating a log of its latest interactions (e.g. cookies, JavaScript, embedded HTTP objects). The crawling plugin running on the mobile device sends the interaction log corresponding to the website being visited in the form of SQL statements to the crawling backend running on a server. This way, the amount of state kept in the mobile device’s main memory is minimal and the crawl database, which can be several Megabytes in size, is generated by the supporting server's side. 

\begin{figure}[h] 
\centering \includegraphics[scale=0.70]{diagrams/component_diagram.png}
\caption{Prototype's Runtime Interactions.}
\label{fig:component_diagram}
\end{figure}


\begin{figure*}[ht] 
\centering \includegraphics[scale=0.70]{diagrams/db_diagram.png}
\captionof{figure}{FourthParty SQLite database schema}
\label{fig:db_schema}
\end{figure*}

\begin{figure*}[ht] 
\centering \includegraphics[scale=0.30]{diagrams/cookies_cdf.png}
\caption{Cookie size distribution with Top 500 dataset}
\label{fig:cookie_cdf}
\end{figure*}

\begin{figure*}[ht] 
\centering \includegraphics[scale=0.30]{diagrams/javascript_cdf.png}
\caption{JavaScript script size distribution with Top 500 dataset}
\label{fig:cookie_cdf}
\end{figure*}

\subsection{Prototype}
We took advantage of the fact that the FourthParty\footnote{http://www.fourthparty.info} project is open-source. After analyzing its codebase, we ported its core functionality over to support Android-based mobile devices, such as smartphones and tablets. FourthPartyMobile is implemented in Java and JavaScript, leveraging both the Android SDK and the Mozilla Add-On SDK. Persistent storage is fully compliant with FourthParty's SQLite database schema. Thus, we provide a standardized representation for traditional and mobile crawls, which facilitates data analysis. Our Crawling Backend is written in java with a SQLite JDBC library that supports Mac OS, Linux and Windows, so it should be fully multi-platform. It also supports concurrency, so multiple crawls can be recorded simultaneously. 

\section{Methodology}
DIEGO

\subsection{Automated Crawls}

We tried automating crawls on Firefox Mobile with MozMill, Selenium, Scriptish and Robocop without any success. The few frameworks that are compatible with Android do not support Firefox Mobile (fennec) – they only interact with the lackluster Android Web Browser. To the best of our knowledge (hours worth of Google searches), it seems that there are no testing frameworks out there that can automate Firefox Mobile crawls. Even the browser plugins that do this on the desktop version (e.g. Flem) have not been ported to the Mozilla Mobile SDK. Bare in mind that Mozilla Mobile SDK was released on February 21 of 2012 (Announcing Add-on SDK 1.5!), so it needs more time to mature.

Our solution was to use JavaScript code on one site to trigger the crawl on a separate tab. We successfully tested this method on all platforms (i.e. desktop, tablet, smartphone). This, we automated the generation of the HTML website containing the JavaScript code to facilitate the creation of these scripts for arbitrary URL lists. A timeout of 30 seconds was used for every one of our crawls, meaning that a new URL was visited every 30 seconds.

\subsection{URL Datasets}

We crawled two URL datasets parsed from \emph{Alexa - Top Sites in United States} on January 7, 2013: (1) Top 100 and (2) Top 500.

After a series of crashes caused by websites containing non-western character sets, we made the decision to focus on the Top US Sites instead of the Top Global Sites.

\subsection{Devices}

All crawls were conducted between January 7, 2013 and January 10, 2013. Six different devices were used to go through the two URL datasets, for a total of 12 result sets.

\begin{itemize}
\item Desktop (Ubuntu 12.04, Firefox 11.0)

\item Asus Transformer Pad TF300T (10.1-inch Tablet)

\item Samsung Galaxy Tab 2 (7.0-inch Tablet)

\item Emulated Nexus 7 (7.0-inch Tablet)

\item HTC Evo 4G (4.8-inch Smartphone)

\item Emulated Nexus S (4.0-inch Smartphone)
\end{itemize}

\subsection{FourthParty SQLite Database Schema}
Figure \ref{fig:db_schema} shows the database schema generated on every crawl.

\section{Data Analysis}

\subsection{Main Players}
CHRIS

Three types of players: advertisers, content providers, and third-party content providers (embedded in sites)

\subsection{Cookie/JavaScript Pervasiveness}
CHRIS

By website category (e.g. porn, news, etc) and by domain (e.g. com, net, etc)

\subsection{Desktop vs Mobile Tracking}
ALL OF US

\subsection{Physical vs Emulated Devices}
DIEGO

\subsubsection{Cookie Interactions}
\begin{table}[htbp]
  \centering
  \caption{Cookie action counts with Top 100 dataset}
    \begin{tabular}{|c|c|c|c|}
    \hline
    \multicolumn{1}{|c|}{\multirow{2}[4]{*}{\textbf{Device}}} & \multicolumn{3}{|c|}{\textbf{Cookies - Top 100 Dataset}} \\ \cline{2-4}
    \multicolumn{1}{|c|}{} & \multicolumn{1}{|c|}{\textbf{Added}} & \multicolumn{1}{|c|}{\textbf{Changed}} & \multicolumn{1}{|c|}{\textbf{Deleted}} \\ \hline
    \multicolumn{1}{|l|}{\textbf{HTC Evo 4G}} & 936   & 1214  & 18 \\
    \multicolumn{1}{|l|}{\textbf{Emulated Nexus S}} & 929   & 1130  & 15 \\
    \multicolumn{1}{|l|}{\textbf{Asus Pad TF300T}} & 1093  & 1351  & 40 \\
    \multicolumn{1}{|l|}{\textbf{Samsung G. Tab 2}} & 1374  & 2494  & 107 \\
    \multicolumn{1}{|l|}{\textbf{Emulated Nexus 7}} & 1282  & 2029  & 97 \\ \hline
    \end{tabular}%
  \label{tab:addlabel}%
\end{table}%

\begin{table}[htbp]
  \centering
  \caption{Cookie action counts with Top 500 dataset}
    \begin{tabular}{|c|c|c|c|}
    \hline
    \multicolumn{1}{|c|}{\multirow{2}[4]{*}{\textbf{Device}}} & \multicolumn{3}{|c|}{\textbf{Cookies - Top 500 Dataset}} \\ \cline{2-4}
    \multicolumn{1}{|c|}{} & \multicolumn{1}{|c|}{\textbf{Added}} & \multicolumn{1}{|c|}{\textbf{Changed}} & \multicolumn{1}{|c|}{\textbf{Deleted}} \\ \hline
    \multicolumn{1}{|l|}{\textbf{HTC Evo 4G}} & 4387  & 4545  & 101 \\
    \multicolumn{1}{|l|}{\textbf{Emulated Nexus S}} & 4665  & 4946  & 100 \\
    \multicolumn{1}{|l|}{\textbf{Asus Pad TF300T}} & 5163  & 6190  & 205 \\
    \multicolumn{1}{|l|}{\textbf{Samsung G. Tab 2}} & 5630  & 7501  & 319 \\
    \multicolumn{1}{|l|}{\textbf{Emulated Nexus 7}} & 4501  & 5107  & 116 \\ \hline
    \end{tabular}%
  \label{tab:addlabel}%
\end{table}%

\begin{table}[htbp]
  \centering
  \caption{Emulated vs Physical deltas in terms of cookie action counts}
    \begin{tabular}{|c|c|c|c|c|}
    \hline
    \multicolumn{1}{|c|}{\multirow{2}[4]{*}{\textbf{Device}}} & \multicolumn{1}{|c|}{\multirow{2}[4]{*}{\textbf{Dataset}}} & \multicolumn{3}{|c|}{\textbf{Cookies}} \\ \cline{3-5}
    \multicolumn{1}{|c|}{} & \multicolumn{1}{|c|}{} & \multicolumn{1}{|c|}{\textbf{Added}} & \multicolumn{1}{|c|}{\textbf{Changed}} & \multicolumn{1}{|c|}{\textbf{Deleted}} \\ \hline
    \multicolumn{1}{|l|}{\multirow{2}[4]{*}{\textbf{Phone}}} & Top 100 & -0.75\% & -6.92\% & -16.67\% \\
    \multicolumn{1}{|l|}{} & Top 500 & 6.34\% & 8.82\% & -0.99\% \\
    \hline
    \multicolumn{1}{|l|}{\multirow{2}[4]{*}{\textbf{Tablet}}} & Top 100 & 3.93\% & 5.54\% & 31.97\% \\
    \multicolumn{1}{|l|}{} & Top 500 & -16.59\% & -25.40\% & -55.73\% \\ \hline
    \end{tabular}%
  \label{tab:addlabel}%
\end{table}%


\begin{table}[htbp]
  \centering
  \caption{Cookie longevity with Top 500 dataset}
    \begin{tabular}{|c|c|c|c|c|}
    \hline
    \multicolumn{1}{|c|}{\multirow{2}[4]{*}{\textbf{Device}}} & \multicolumn{4}{|c|}{\textbf{Cookies - Top 500 Dataset}} \\ \cline{2-5}
    \multicolumn{1}{|c|}{} & \textbf{Perm.} & \textbf{Temp.} & \textbf{\% Perm.} & \textbf{Total} \\ \hline
    \multicolumn{1}{|l|}{\textbf{HTC Evo 4G}} & 2805  & 6228  & 31.05\% & 9033 \\
    \multicolumn{1}{|l|}{\textbf{Emulated Nexus S}} & 3002  & 6709  & 30.91\% & 9711 \\
    \multicolumn{1}{|l|}{\textbf{Asus Pad TF300T}} & 3364  & 8194  & 29.11\% & 11558 \\
    \multicolumn{1}{|l|}{\textbf{Samsung G. Tab 2}} & 3779  & 9671  & 28.10\% & 13450 \\
    \multicolumn{1}{|l|}{\textbf{Emulated Nexus 7}} & 3014  & 6710  & 31.00\% & 9724 \\ \hline
    \end{tabular}%
  \label{tab:addlabel}%
\end{table}%

\begin{table}[htbp]
  \centering
  \caption{Cookie size distribution with Top 500 dataset}
    \begin{tabular}{|c|c|c|}
    \hline
    \multicolumn{1}{|c|}{\multirow{2}[4]{*}{\textbf{Device}}} & \multicolumn{2}{|c|}{\textbf{Cookies - Top 500 Dataset}} \\ \cline{2-3}
    \multicolumn{1}{|c|}{} & \textbf{Avg Size (Bytes)} & \textbf{Best CDF Fit} \\ \hline
    \multicolumn{1}{|l|}{\textbf{HTC Evo 4G}} & 54.29 & Exp., $\lambda$=0.01842 \\
    \multicolumn{1}{|l|}{\textbf{Emulated Nexus S}} & 51.86 & Exp., $\lambda$=0.01928 \\
    \multicolumn{1}{|l|}{\textbf{Asus Pad TF300T}} & 71.37 & Exp., $\lambda$=0.01401 \\
    \multicolumn{1}{|l|}{\textbf{Samsung G. Tab 2}} & 85.37 & Exp., $\lambda$=0.01171 \\
    \multicolumn{1}{|l|}{\textbf{Emulated Nexus 7}} & 59.20 & Exp., $\lambda$=0.01689 \\ \hline
    \end{tabular}%
  \label{tab:addlabel}%
\end{table}%

\subsubsection{HTTP Interactions}

\begin{table}[htbp]
  \centering
  \caption{HTTP Responses by content type, Top 100 dataset}
    \begin{tabular}{|c|c|c|c|c|c|}
    \hline
    \multicolumn{1}{|c|}{\multirow{2}[4]{*}{\textbf{Device}}} & \multicolumn{4}{|c|}{\textbf{HTTP Responses - Top 100}} \\ \cline{2-5}
    \multicolumn{1}{|c|}{} & \textbf{total} & \textbf{image} & \textbf{JS} & \textbf{other} \\ \hline
    \multicolumn{1}{|l|}{\textbf{HTC Evo 4G}} & 3710  & 2269  & 721   & 720 \\
    \multicolumn{1}{|l|}{\textbf{Emulated Nexus S}} & 3546  & 2208  & 678   & 660 \\
    \multicolumn{1}{|l|}{\textbf{Asus Pad TF300T}} & 4440  & 2708  & 914   & 818 \\
    \multicolumn{1}{|l|}{\textbf{Samsung G. Tab 2}} & 5460  & 3069  & 1294  & 1097 \\
    \multicolumn{1}{|l|}{\textbf{Emulated Nexus 7}} & 4743  & 2848  & 1007  & 888 \\ \hline
    \end{tabular}%
  \label{tab:addlabel}%
\end{table}%

\begin{table}[htbp]
  \centering
  \caption{HTTP Responses by content type, Top 500 dataset}
    \begin{tabular}{|c|c|c|c|c|c|}
    \hline
    \multicolumn{1}{|c|}{\multirow{2}[4]{*}{\textbf{Device}}} & \multicolumn{4}{|c|}{\textbf{HTTP Responses - Top 500}} \\ \cline{2-5}
    \multicolumn{1}{|c|}{} & \textbf{total} & \textbf{image} & \textbf{JS} & \textbf{other} \\ \hline
    \multicolumn{1}{|l|}{\textbf{HTC Evo 4G}} & 17848 & 10932 & 3616  & 3300 \\
    \multicolumn{1}{|l|}{\textbf{Emulated Nexus S}} & 18376 & 11052 & 3870  & 3454 \\
    \multicolumn{1}{|l|}{\textbf{Asus Pad TF300T}} & 24283 & 15045 & 5131  & 4107 \\
    \multicolumn{1}{|l|}{\textbf{Samsung G. Tab 2}} & 27544 & 16664 & 6132  & 4748 \\
    \multicolumn{1}{|l|}{\textbf{Emulated Nexus 7}} & 20441 & 12735 & 4298  & 3408 \\
 \hline
    \end{tabular}%
  \label{tab:addlabel}%
\end{table}%

\begin{table}[htbp]
  \centering
  \caption{Emulated vs Physical deltas in terms of HTTP response content types}
    \begin{tabular}{|c|c|c|c|c|c|}
    \hline
    \multicolumn{1}{|c|}{\multirow{2}[4]{*}{\textbf{Device}}} & \multicolumn{1}{|c|}{\multirow{2}[4]{*}{\textbf{Dataset}}} & \multicolumn{4}{|c|}{\textbf{HTTP Responses}} \\ \cline{3-6}
    \multicolumn{1}{|c|}{} & \multicolumn{1}{|c|}{} & \multicolumn{1}{|c|}{\textbf{total}} & \multicolumn{1}{|c|}{\textbf{image}} & \multicolumn{1}{|c|}{\textbf{JS}} & \multicolumn{1}{|c|}{\textbf{other}} \\ \hline
    \multicolumn{1}{|l|}{\multirow{2}[4]{*}{\textbf{Phone}}} & Top 100 & -4.42\% & -2.69\% & -5.96\% & -8.33\% \\
    \multicolumn{1}{|l|}{} & Top 500 & 2.96\% & 1.10\% & 7.02\% & 4.67\% \\
    \multicolumn{1}{|l|}{\multirow{2}[4]{*}{\textbf{Tablet}}} & Top 100 & -4.18\% & -1.40\% & -8.79\% & -7.26\% \\
    \multicolumn{1}{|l|}{} & Top 500 & -21.12\% & -19.68\% & -23.68\% & -23.03\% \\ \hline
    \end{tabular}%
  \label{tab:addlabel}%
\end{table}%

\subsubsection{JavaScript Interactions}

\begin{table}[htbp]
  \centering
  \caption{JavaScript script size distribution, Top 500 dataset}
    \begin{tabular}{|c|c|p{2.3cm}|}
    \hline
    \multicolumn{1}{|c|}{\multirow{2}[4]{*}{\textbf{Device}}} & \multicolumn{2}{|c|}{\textbf{JavaScript - Top 500}} \\ \cline{2-3}
    \multicolumn{1}{|c|}{} & \textbf{Avg Size (B)} & \textbf{Best CDF Fit} \\ \hline
    \multicolumn{1}{|l|}{\textbf{HTC Evo 4G}} & 14,647.11 & \parbox{4cm}{Logistic,\\$\alpha$=14,647.11,\\$\lambda$=0.00007\\} \\ 
    \multicolumn{1}{|l|}{\textbf{Emulated Nexus S}} & 15,107.96 & \parbox{4cm}{Logistic,\\ $\alpha$=15,107.96,\\$\lambda$=0.00006\\} \\
    \multicolumn{1}{|l|}{\textbf{Asus Pad TF300T}} & 13,708.73 & \parbox{4cm}{Logistic,\\ $\alpha$=13,708.73,\\$\lambda$=0.00007\\} \\
    \multicolumn{1}{|l|}{\textbf{Samsung G. Tab 2}} & 12,711.00 & \parbox{4cm}{Logistic,\\ $\alpha$=12,711.00,\\$\lambda$=0.00007\\} \\
    \multicolumn{1}{|l|}{\textbf{Emulated Nexus 7}} & 14,445.22 & \parbox{4cm}{Logistic,\\ $\alpha$=14,445.22,\\$\lambda$=0.00006\\} \\ \hline
    \end{tabular}%
  \label{tab:addlabel}%
\end{table}%

\subsection{Privacy Policy Case Study}
After running our web crawls, we designed a case study in which we look at the privacy policies and related data collected from our crawls of three categories of websites that we found were amongst the most popular sites today: social networks, news sites, and e-commerce sites. While pornography sites are also amongst the most popular sites online, we do not include this fourth category in our case study for reasons of decency. 

Based on the list of the Alexa Top 100 US visited websites, we chose three websites amongst the top 25 most visited sites, one representing each of our three categories. Our case study examines the privacy policies of LinkedIn\footnote{http://www.linkedin.com} (social network), CNN\footnote{http://www.cnn.com} (news) and Amazon\footnote{http://www.amazon.com} (e-commerce).

The case study has five stages, each stage looking at the privacy policies in more detail:
\begin{enumerate}
\item Compare the contents of the privacy policies displayed when visiting each of the sites on our three platforms (desktop, tablet, and smartphone).
\item Compare the length of the privacy policies of each website.
\item Examine the topics covered, i.e., the sections included in the policies.
\item Inspect some of the language used in select sections of the policies.
\item Compare the presented cookie policies with the collected web crawl data.
\end{enumerate}

The first question we wanted to answer in our case study was \emph{Do privacy policies vary between normal web and mobile web?} Thus, the first stage of the case study, seeks to answer this question via a very simple method. We visited the three websites on our three platforms and downloaded the presented privacy policies for each site on each platform. This gave us three privacy policies each for LinkedIn, CNN, and Amazon. 

The easiest and quickest way to determine if the contents of the policies differ for a single website was to run the \texttt{diff} command between the policies collected for each pair of platforms\footnote{The policies were all saved as text documents, and stripped of tabs and empty lines via a python script.}: desktop - smartphone, desktop - tablet, and tablet - smartphone. Taking the more pessimistic standpoint, we were expecting to find that a single website's policies would differ more between the three platforms. But what we discovered in reality is that in the cases of LinkedIn and CNN, the privacy policies for all three platforms are exactly identical. \textbf{keep this}We consider this to be a positive finding as this proves that the websites make an effort to maintain consistency across the devices.\textbf{??} In the case of Amazon, the privacy policies presented when visiting the website through a desktop and a tablet are exactly identical. However, the smartphone version of the policy is much shorter and refers the reader to the full (desktop) version of their privacy notice. We would like to note that the fact that the privacy policies were basically all identical for each of the three websites in our case study greatly facilitated our further analysis of the privacy policies. Any further examination was done using the desktop version of the policies.

Our next point of study was the length of each of the privacy policies as this is an indicator of the amount of detail the websites offer their visitors. Running the \texttt{wc -w} command on each website's policy. The the resulting word counts are summarized in Table \ref{tab:wc}.

\begin{table}[htbp]
  \centering
  \caption{Word counts of the privacy policies of our three studied websites.}
    \begin{tabular}{|c|c|}
    \hline
    \textbf{Website} & \textbf{Word Count} \\    
     \hline
     LinkedIn & 6324 \\
     CNN & 2734 \\
     Amazon & 2691 \\
     \hline
    \end{tabular}%
  \label{tab:wc}%
\end{table}%

We were not surprised to see that the social network has the longest privacy policy, as these types of websites are in a constant race to see who can protect their users' privacy better. However, we were also expecting the e-commerce site to have a longer privacy policy, given that users purchase items through these sites by inputting sensitive data such as credit card numbers and home addresses. Thus, we were expecting to see that the website would supply the user with more detailed information about their privacy practices to reassure them that especially their sensitive data will be dealt with appropriately. What we were not expecting was to find that Amazon would have the shortest policy of our three studied websites.

In the end, we can say that LinkedIn's meticulously written privacy policy is the most informative for various kinds of users. Their privacy summary is an incredibly helpful tool, especially for those users who want to be informed about what their social network is doing with respect to their data and privacy, but do not or cannot take the time to read through the entire document. In addition, the fact that LinkedIn also includes a separate Cookie Policy shows that they put the time into creating documents that would fully inform their users.

\section{Conclusions and Future Work}

\nocite{*}
\bibliographystyle{abbrv}
\bibliography{sigproc}
\end{document}
